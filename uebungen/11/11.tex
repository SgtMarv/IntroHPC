\documentclass[a4paper,11pt]{scrartcl}

\usepackage[ngerman]{babel} 
\usepackage[T1]{fontenc}
\usepackage[utf8]{inputenc}
\usepackage{hyperref}

\usepackage{amsmath,amsfonts,amssymb}
\usepackage{graphicx}
\usepackage{siunitx}

\usepackage{epstopdf}

\setcounter{section}{11}


\begin{document}
\hfill Alexander Schnapp

\hfill Max Menges

\hfill introhpc02

\begin{center}
\underline{\Huge{Intro HPC: Blatt 11}}\\
\large{26.1.2015}\\
\end{center}


\subsection{Reading}
\subsubsection{On Achieving High Message Rates}
I
\subsubsection{Global GPU Address Spaces for Efficient
Communication in Heterogeneous Clusters}
In this paper, the authors propose and implement a model for direct GPU to GPU message passing, by-passing the CPU, called GGAS -- Global GPU Adress Space. The approach uses a shared memory engine to map some GPU registers to a cluster wide global memory. For testing, acustom network device was implemented on an FPGA. 

The CPU is now no longer required to initiate communication actions and can be utilized for other actions. On a test implementation using two nodes, the authors ran several benchmarks including latency, bandwith and running a stencil code and compared their results to the performance of an Infiniband network with traditional communication methods. First results show a speedup in various performed tasks.

The paper describes what seems to be a novel idea for GPU communications. A follow up paper for extended measurements (better network device, scalability, etc.) would be interesting. Measurements seem a bit preliminary and the the technical implementation could have been a bit more detailed, otherwise a good paper.

\end{document}
