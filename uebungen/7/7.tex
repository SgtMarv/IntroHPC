\documentclass[a4paper,11pt]{scrartcl}

\usepackage[ngerman]{babel} 
\usepackage[T1]{fontenc}
\usepackage[utf8]{inputenc}
\usepackage{hyperref}

\usepackage{amsmath,amsfonts,amssymb}
\usepackage{graphicx}
\usepackage{siunitx}

\usepackage{epstopdf}

\setcounter{section}{5}


\begin{document}
\hfill Alexander Schnapp

\hfill Max Menges

\hfill introhpc02

\begin{center}
\underline{\Huge{Intro HPC: Blatt 7}}\\
\large{24.11.1014}\\
\end{center}

\subsection{Reading:Roofline - An Insightful
Visual
Performance
Model for
Multicore
Architectures}



\subsection{Reading: LogP: Towards a Realistic Model of Parallel Computation}
The author is intruducing a model, that
discribe the bottlenegs of a distributed memory
multiprocessor in which processors communicate by point-to-
point messages.
Therefore it is based on a few of parameters like latency (L), overhead (o) , bandwidth (g) of communication and the number of processes (P) and the assumption of a finite capacity.
Furthermore it is generall for different types ofcommunication protocolls or applications. 

The author is finding, that for some applications some parameters can get negligible, which makes a simplification of the model possible. He testeted the model on different workloads like the FFT and the LU-Decomposition, to show how the use of the model can lead to effecient parallel applications in practice.
In the paper he is also comparing this model to the widely used PRAM and the BSP model, which do not
accurately reflect the performance characteristics of suchy systems, in his opinion.

This model is promissing a couple of advantages to other often used models like including assynchrounus algorithms. It is 
 very generall, so it might give a good overview of a system but not very detailed.
\end{document}
